\chapter{Einführung}

\section{Übersicht über die Applikation}
Bei dieser Applikation handelt es sich um einen
\textit{Single Player Dungeon Crawler}, der
\textit{The Binding Of Isaac} in seinen Grundzügen nachempfunden ist.
Der Spieler bewegt sich durch prozedural generierte Level. Dabei kann
er verschiedene Gegenstände einsammeln wie Rüstung (\textit{Armor}),
Waffen (\textit{Weapons}), Münzen (\textit{Coins}) und Herzen
(\textit{Hearts}). Diese Gegenstände sollen ihm dabei helfen zu
überleben. Die Münzen stellen jedoch den Punktestand dar.
Während der Spieler sich durch die Räume bewegt, stößt er
auf eine Palette unterschiedlicher Gegner (\textit{Spider}, 
\textit{Skeleton}, {Zombie}, \textit{Ogre}). Diese besitzen auch
unterschiedliche Fähigkeiten und versuchen den Spieler anzugreifen.
Betritt der Spieler einen Raum, so wird dieser gesperrt, bis er alle
darin befindlichen Gegner besiegt hat. Erst dann öffnen sich die Türen
wieder und der Spieler kann in anliegende Räume laufen.
Stirbt der Spieler dabei, so ist das Spiel zu Ende. Bleibt er am Leben,
hangelt er sich von Level zu Level. Am Ende eines jeden Levels steht
ein Raum mit einer einzigen Falltür in der Mitte, welche den Spieler
in das nächste Level bringt.

Das Spiel ist rundenbasiert. In jeder Runde stehen Aktionen zur 
Verfügung, wie etwa Angreifen (\textit{Space}), Bewegen
(\textit{W}, \textit{A}, \textit{S} oder \textit{D}) oder Aufheben
(\textit{E}). Auch die Gegner agieren rundenbasiert. Diese können
jedoch lediglich angreifen. Sie laufen dem Spieler mit einer Runde
Zeitversatz hinterher. Dabei können Gegner sich auch gegenseitig den
Weg blockieren.

Damit der Spielstand in Form von Level, Gegnern und Items nachvollziehbar
bleibt, gibt es eine einfache Anzeige, welche mithilfe von ANSI- und
ASCII-Zeichen den Spielinhalt darstellt. Dabei helfen unterschiedliche
Symbole und Farben den Inhalt zu verstehen.

\section{Wie startet man die Applikation?}

\section{Wie testet man die Applikation?}
