\chapter{Weitere Prinzipien}

\section{Analyse GRASP: Geringe Kopplung}
\subsection*{Positiv-Beispiel}
\vspace{0.5cm}
\begin{figure}[H]
    \centering
    \includegraphics[width=1\linewidth]{Bilder/Visualisierung/IInteractionHandler_structure.png}
    \caption{Analyse GRASP: Geringe Kopplung / Positiv}
\end{figure}

\subsection*{Negativ-Beispiel}
\vspace{0.5cm}
\begin{figure}[H]
    \centering
    \includegraphics[width=1\linewidth]{Bilder/Visualisierung/RoomPositionNoInterface_structure.png}
    \caption{Analyse GRASP: Geringe Kopplung / Negativ}
\end{figure}

\iffalse
Um die beiden Klassen RoomPosition und Room besser zu entkoppeln, können wir die direkte Abhängigkeit der RoomPosition von der konkreten Room-Implementierung lösen und stattdessen eine abstrakte Schnittstelle oder eine abstrakte Basisklasse einführen. Dadurch wird die Abhängigkeit lockerer und die Klassen können unabhängig voneinander entwickelt, getestet und gewartet werden.

Hier ist ein Beispiel, wie die Klassen entkoppelt werden könnten:

java
Copy code
public interface Room {
    boolean contains(Position position);
    boolean isBlockedByEnemy(Position position);
}

public class ConcreteRoom implements Room {
    // Implementierung der Methoden für den konkreten Raum
    // ...
}

public class RoomPosition extends Position implements IVectorizable {
    private final Room room;

    public RoomPosition(int x, int y, Room room) {
        super(x, y);
        this.room = room;
    }

    public Room getRoom() {
        return this.room;
    }

    public boolean isInside() {
        return this.room.contains(this);
    }

    public boolean isBlockedByEnemy() {
        return this.room.isBlockedByEnemy(this);
    }

    // Restliche Methoden und Implementierung der Klasse
    // ...
}
In diesem Beispiel haben wir die Schnittstelle Room eingeführt, die die benötigten Methoden definiert, die die RoomPosition verwenden möchte. Die konkrete Implementierung ConcreteRoom implementiert diese Schnittstelle und enthält die spezifische Logik für den Raum.

Durch die Verwendung der Schnittstelle Room anstelle der konkreten Room-Klasse wird die Abhängigkeit in der RoomPosition gelockert. Jetzt kann die RoomPosition mit jeder Klasse arbeiten, die die Room-Schnittstelle implementiert, unabhängig von ihrer konkreten Implementierung.

Dies ermöglicht es uns, verschiedene Arten von Räumen zu erstellen und auszutauschen, ohne dass Änderungen an der RoomPosition erforderlich sind. Außerdem erleichtert es das Testen der RoomPosition, da wir Mock-Implementierungen der Room-Schnittstelle verwenden können, um isolierte Tests durchzuführen.

Die Entkopplung der Klassen erhöht die Flexibilität, Wartbarkeit und Testbarkeit des Codes und fördert eine bessere Modulbildung und Wiederverwendbarkeit der Komponenten.

\fi

\section{Analyse GRASP: Hohe Kohäsion}
Das nachfolgende UML-Diagramm zeigt die Klasse \textit{Buffer}, welche
eine sehr hohe Kohäsion aufweist. Sie beschreibt einen Puffer für
Anzeigedaten. Der Puffer enthält sowohl Farb- als auch Zeichendaten
und besitzt eine Größendefinition.

Die Kohäsion ist sehr hoch, weil die Klasse eine einzige
Verantwortlichkeit hat, nämlich die Verwaltung des Puffers. Die
enthaltenen Funktionen sind darüber hinaus lediglich für die 
Manipulation und Anzeige des Pufferinhalts verantwortlich. Sie bilden
verschiedene Manipulationsmethoden ab. Zudem existiert keine externe
Abhängigkeit. Die komplette Logik für die Klasse ist in ihr vereint.

\vspace{0.5cm}
\begin{figure}[H]
    \centering
    \includegraphics[width=0.75\linewidth]{Bilder/Visualisierung/Buffer_structure.png}
    \caption{Analyse GRASP: Hohe Kohäsion}
\end{figure}

\iffalse
Die gegebene Klasse "Buffer" erfüllt das Kriterium der hohen Kohäsion,
da sie eine klare und spezifische Aufgabe erfüllt, nämlich die
Verwaltung eines Puffers für die Darstellung von Zeichen und Farben.
Hier sind einige Gründe, warum sie hohe Kohäsion aufweist:

Zusammengehörigkeit: Die Methoden und Eigenschaften der Klasse dienen
alle dem Zweck der Pufferverwaltung. Sie ermöglichen das Setzen von
Zeichen und Farben, das Hinzufügen von Inhalten, das Erzeugen von
speziellen Puffern (z. B. mit Rahmen) und die Darstellung des Puffers.

Klare Verantwortlichkeiten: Die Klasse konzentriert sich ausschließlich
auf die Pufferverwaltung und enthält keine Funktionalitäten, die nicht
damit verbunden sind. Sie bietet Methoden zum Setzen von Zeichen,
Farben, zum Schreiben von Inhalten, zum Erzeugen spezieller Puffer und
zur Darstellung des Puffers.

Geringe Abhängigkeit von externen Komponenten: Die Klasse hat keine
Abhängigkeiten von anderen Klassen oder externen Ressourcen. Sie
enthält alle erforderlichen Methoden und Datenstrukturen, um die
Pufferverwaltung durchzuführen.

Einfache Wiederverwendbarkeit: Der Buffer kann in verschiedenen
Anwendungen oder Modulen wiederverwendet werden, in denen eine
Pufferverwaltung erforderlich ist. Die Klasse ist gut isoliert und
unabhängig von anderen Teilen des Systems.

Insgesamt erfüllt die Klasse "Buffer" das Prinzip der hohen Kohäsion,
indem sie eine klare und spezifische Aufgabe erfüllt und ihre Methoden
und Eigenschaften darauf ausgerichtet sind. Dies erleichtert die
Lesbarkeit, Wartbarkeit und Wiederverwendbarkeit des Codes.
\fi

\section{Don't Repeat Yourself (DRY)}

Der erste Code zeigt den Zustand vor der Klassen \textit{Entity},
\textit{Player} und \textit{Goblin} vor der Code-Deduplikation 
(Commit 3d905e1). Tatsächlich waren allerdings noch einige mehr
Klassen auf ähnliche Weise betroffen.

Zu sehen ist, dass die Konstruktor-Funktion von \textit{Entity} nicht
die Möglichkeit bietet alle Felder zu setzen. Diese sind jedoch wichtig
bei der Definition von konkreten Entities, wie \textit{Goblin} oder
\textit{Player}. Daher muss jede Klasse, welche von \textit{Entity}
erbt, die Felder \textit{weapon} und \textit{position} manuell
gesetzt bekommen.

Der zweite Code zeigt, dass die Konstruktor-Funktion entsprechend
erweitert wurde, sodass die Felder nun auch als Argumente entgegengenommen
und an einer zentralen Stelle (\textit{Entity}) entsprechend gesetzt
werden (Commit b125cf0). Damit ist \textit{DRY} erfüllt. Statt
mehrfacher gleicherartiger Wertezuweisungen der Felder in allen
Kindsklassen von \textit{Entity}, geschieht die Zuweisung nur noch an
einer Stelle. 

Zum einen verbessert dies die Komplexität und Wartbarkeit, weil nicht
mehr mehrere Stellen geändert werden müssen, sollte sich z.B. der
Name eines betroffenen Feldes ändern. Zum anderen ist es lesbarer und
weniger fehleranfällig.

\lstset{ 
    language=Java,
    basicstyle=\fontfamily{pcr}\selectfont\footnotesize,
    keywordstyle=\color{purple},
    numberstyle=\tiny,    
    showspaces=false,
    showstringspaces=false,
    showtabs=false, 
    frame=single,
    tabsize=2,
    rulesepcolor=\color{gray},
    rulecolor=\color{black},
    captionpos=b,
    breaklines=true,
    breakatwhitespace=false, 
}

\vspace{0.5cm}
\begin{lstlisting}[caption={Don't Repeat Yourself (Vorher)}]
    public abstract class Entity {
        protected String name;
        protected int health;
        protected Position position;
        protected int baseArmor;
        protected int strength;
        protected Weapon weapon;
     
     
        public Entity(int health, int baseArmor, int strength, String name){
            this.health = health;
            this.baseArmor = baseArmor;
            this.strength = strength;
            this.name = name;
        }
        
        ...
    }     

    public class Player extends Entity {

        public Player(String name) {
            super(100, 5, 10, name);
            weapon = Weapon.FISTS;
            position = new Position(0, 0);
        }


        public String getName() {
            return name;
        }

    }

    public class Goblin extends Entity {
        public Goblin(Position position) {
            super(10, 2, 5, "Goblin");
            weapon = Weapon.DAGGER;
            this.position = position;
        }
    }
\end{lstlisting}

\vspace{0.5cm}
\begin{lstlisting}[caption={Don't Repeat Yourself (Nachher)}]
    public abstract class Entity {
        protected String name;
        protected int health;
        protected Position position;
        protected int baseArmor;
        protected int strength;
        protected Weapon weapon;
     
        public Entity(int health, int baseArmor, int strength, String name, Weapon weapon, Position position){
            this.health = health;
            this.baseArmor = baseArmor;
            this.strength = strength;
            this.name = name;
            this.weapon = weapon;
            this.position = position;
        }
     
        ...
    }

    public class Player extends Entity {

        public Player(String name) {
            super(100, 5, 10, name, Weapon.FISTS, new Position(0,0));
        }

        public String getName() {
            return name;
        }

    }

    public class Goblin extends Entity {

        public Goblin(Position position) {
            super(10, 2, 5, "Goblin", Weapon.DAGGER, position);
        }

    }

\end{lstlisting}