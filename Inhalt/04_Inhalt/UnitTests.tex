\chapter{Unit Tests}

\section{10 Unit Tests}

\hyphenation{DirectionTest}
\begin{table}[H]
    \centering
    \begin{tabular}{|p{8cm}|p{8cm}|}
      \hline
      \textbf{Unit Test} & \textbf{Beschreibung} \\
      \hline
      1. VectorTest\#dotTest & Testet \textit{dotTest}-Funktion der Klasse \textit{Vector} darauf, ob sie korrekt Skalarprodukte berechnet \\
      \hline
      2. VectorTest\#lengthTest & Testet \textit{length}-Funktion der Klasse \textit{Vector} darauf, ob sie korrekt Vektorlängen berechnet \\
      \hline
      3. VectorTest\#getClockwiseAngleFromTest & Testet \textit{getClockwiseAngleFrom}-Funktion der Klasse \textit{Vector} darauf, ob sie korrekt Winkel zwischen zwei Vektoren berechnet \\
      \hline
      4. NumericTest\#clampTest & Testet \textit{clamp}-Funktion der Klasse \textit{Numeric} darauf, ob sie korrekt Werte in einem gegebenen Interval einschließt \\
      \hline
      5. EntityTest\#getPreferredMovement DirectionTest & Testet \textit{getPreferredMovementDirection}-Funktion der Klasse \textit{Enemy} darauf, ob sie korrekt eine Bewegung aufgrund der Position des Spielers wählt \\
      \hline
      6. RoomGridTest\#fitsTest & Testet \textit{fits}-Funktion der Klasse \textit{RoomGrid} darauf, ob sie korrekt feststellen kann, ob ein Raum noch auf die Spielkarte passt \\
      \hline
      7. RoomPositionTest\#getMaxDistanceAlongAnyAxisTest & Testet \textit{getMaxDistanceAlongAnyAxis}-Funktion der Klasse \textit{RoomPosition} darauf, ob sie korrekt die maximale Entfernung entlang einer von beiden Axen von einem Punkt zum anderen berechnen kann \\
      \hline
      8. CollectionSelectorTest\#randomSubset Test & Testet \textit{random}-Funktion der Klasse \textit{CollectionSelector} darauf, ob sie korrekt Teilmengen der Auswahl selektiert \\
      \hline
    \end{tabular}
    \caption{Unit Tests mit Beschreibung I}
\end{table}

\begin{table}[H]
    \centering
    \begin{tabular}{|p{8cm}|p{8cm}|}
      \hline
      \textbf{Unit Test} & \textbf{Beschreibung} \\
      \hline
      9. GameStateTest\#movementTest & Testet \textit{dotTest}-Funktion der Klasse \textit{Vector} darauf, ob sie korrekt Skalarprodukte berechnet \\
      \hline
      10. VectorTest\#lengthTest & Testet \textit{length}-Funktion der Klasse \textit{Vector} darauf, ob sie korrekt Vektorlängen berechnet \\
      \hline
    \end{tabular}
    \caption{Unit Tests mit Beschreibung II}
\end{table}

\section{ATRIP: Automatic}
Der Begriff \textit{Automatic} bezieht sich auf die automatische
Ausführung von Unit Tests. Dies kann mit herkömmlichen Test-Frameworks
wie etwa \textit{JUnit5} erreicht werden. 

Dabei können Methoden eigens zum Testen erstellt werden, welche mit
\textit{@Test} annotiert werden. Innerhalb der IDE (z.B. IntelliJ)
können sie dann über einen Befehl oder Knopfdruck einzeln oder im
Gesamten ausgeführt werden.

Die Tests sind so konzipiert, dass sie nicht andere Durchläufe beeinflussen.
Sie sind also unabhängig von einander (\textit{Independence}) und 
wiederholbar (\textit{Repeatability}). Das führt zu nachvollziehbaren
Ergebnissen. Zudem besitzen die Tests keine externen Abhängigkeiten.

Zum Schluss gibt JUnit eine Zusammenfassung aus. Diese beschreibt die
erfolgreichen und fehlgeschlagenen Tests.

\section{ATRIP: Thorough}

\section{ATRIP: Professional}

\section{Code Coverage}

\section{Fakes und Mocks}