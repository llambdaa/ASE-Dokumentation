\chapter{Clean Architecture}

\section{Was ist Clean Architecture?}
Die \textit{Clean Architecture} ist ein Gestaltungskonzept, das darauf
abzielt möglichst robuste und wartbare Anwendungen zu entwickeln. Dabei
wird eine Software in hierarchisch mehrere Schichten organisiert, was
gemeinhin als \textit{Onion-Architektur} bekannt ist. Jede Schicht
weist dabei eine spezifische Verantwortung (siehe \textit{domain}, 
\textit{plugins} etc.). 

Die Schichten sind über Schnittstellen verbunden. Hinter den
Schnittstellen steckt dann eine tatsächliche Implementation. Ziel ist
es dadurch eine Trennung zwischen der Logik einer Schicht und den
technischen Details einer anliegenden Schicht zu erzeugen. Statt durch
konkrete Implementierungsdetail sind die Schichten (zumindest von innen
nach außen) über
Funktionsabmachungen in Form der Schnittstellen verbunden. Dadurch
lassen sich leicht Änderungen in einer Schicht vornehmen, ohne dabei
zwangsläufig Änderungen in einer anderen Schicht vornehmen zu müssen.

Entscheidend für das Prinzip ist die Abhängigkeitsrichtung. Äußere
abstraktere Schichten (hierarchisch untergeordnet) können direkt auf
innere Schichten zugreifen und von ihnen abhängig sein. Jedoch gilt
dies nicht für die Umkehrrichtung. Aufrufe von innen nach außen werden
über Schnittstellen und Abstraktionen sichergestellt 
(\textit{Dependency Inversion} und \textit{Injection}).
Tieferliegende Schichten sind zudem am wenigsten abstrakt und damit
am langlebigsten.

Dieser Sachverhalt führt zu Anwendungen, die wartbarer und robuster sind,
weil Änderungen an einer Schicht meist keine Änderungen an anderen
Schichten verursachen. Durch clevere Trennung und Abstraktionen lässt
sich der Anpassungaufwand durch eine beliebige Änderung an vielen Stellen
einsparen.

\section{Analyse der Dependency Rule}

\section{Analyse der Schichten}